\chapter{Conclus�es}
Neste documento apresentamos o projeto WebP2P, uma solu��o para a indisponibilidade tempor�ria de conte�do dos servidores web. Trata-se de montar uma rede de servidores web onde o conte�do dos mesmos � replicado entre eles.

A parte cr�tica do projeto se deu no sentido de como validar o simulador feito pela equipe. A pouca experi�ncia que detinhamos no come�o do projeto fez com que tom�ssemos a decis�o errada ao escolher implementar um simulador pr�prio sem saber como ir�amos valid�-lo. Neste aspecto, o professor Marcos nos orientou nas tomadas de decis�es e de quais estrat�gias seguir para que o trabalho fosse efetuado de maneira satisfat�ria.

O nosso cliente, que por muitas vezes fez papel de orientador devido a natureza do projeto, aparenta estar satisfeito com o trabalho 		desenvolvido. As discuss�es feitas durante o per�odos de reuni�es com Marco foram, em sua maioria, se��es de orienta��o no sentido de como efetuar a pesquisa. Marco nos passou uma experi�ncia que n�o tinhamos ao come�o do projeto. Acreditamos que a sua satisfa��o, al�m de ver a nossa solu��o implementada, est� em ver o nosso crescimento como pesquisadores.

Por ser um trabalho mais relacionado � pesquisa que a desenvolvimento, muitas vezes algumas atividades do o processo de desenvolvimento foi posto em segundo plano. As mudan�as constantes de estrat�gias foram respons�veis por este fator. No nosso trabalho de pesquisa, as abordagens mudavam constantemente, por isso adequar o processo de pesquisa a uma metodologia onde o planejamento de tarefas � bem definido com consider�vel anteced�ncia mostrou-se invi�vel.
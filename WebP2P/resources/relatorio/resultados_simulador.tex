O planejamento inicial da disciplina Projeto-I tinha como uma das user story o desenvolvimento de um prototipo do WebP2P. Durante as reuni�es com o professor Marco e discuss�es com membros do LSD, a equipe chegou a conclus�o de que o desenvolvimentode um simulador seria mais vantajoso para a an�lise do sistema. Fazendo uso do simulador, a equipe foi capaz de analisar o comportamento do sistema em um ambiente configuravel. Com isto, alguns fatores que n�o seriam facilmente controlados, como a banda de um m�quina, poderam ser manipulados.

\subsection{Arquitetura}

Fazendo uso dos conceitos adquiridos na disciplina An�lise de Desempenho de Sistemas Discretos (ADSD), a arquitetura do simulador foi definida como sendo uma rede de filas. Cada entidade do sistema cont�m uma fila de requisi��es a serem processadas, o sistema como um todo � mapeado � uma rede de filas como demonstrado nas figuras~\ref{simulador-figura-top, simulador-figura-queues}. A figura~\ref{simulador-figura-top} demonstra uma vis�o alto n�vel do ambiente simulado, este ambiente segue a arquitetura do sistema descrita anteriormente. Na figura~\ref{simulador-figura-queues}, demonstramos uma vis�o baixno n�vel do simulador, como foi dito cada entidada tem uma fila de requisi��es que ser�o processadas. Ap�s uma requisi��o ser enviada de uma entidade para outra, esta requisi��o tem um atraso at� ser inserida na fila do receptor.

FIGURA AQUI

FIGURA AQUI



\subsection{Valida��o}



\subsection{Desafios}
